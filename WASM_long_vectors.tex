\documentclass[t,aspectratio=169, xcolor={table}]{beamer}
\usepackage[utf8]{inputenc}
\usepackage[russian]{babel}
\usepackage{listings}
\usepackage{hyperref}
\setbeamertemplate{footline}[frame number]
\title{WebAssembly Flexible Vectors Operations\\Развитие концепции кросс-платформ SIMD}
\author{Пётр Пензин}
%\institute{Intel Corporation}
\date{\today}
\begin{document}
\begin{frame}[plain]
  \titlepage
\end{frame}
\begin{frame}
\frametitle{Содержание}
  \begin{itemize}
  \item Архитектура SIMD
  \item SIMD в WebAssembly
  \item Векторные операции для WebAssembly
  \end{itemize}
\end{frame}
\begin{frame}
\frametitle{Архитектура SIMD}
  \begin{itemize}
  \item Многопоточные вычисления -- одна и та же операция выполняется на нескольких элементам одновременно
    % Число потоков постоянно для одного набора комманд, но некоторые позволяют выборочно отключать потоки
  \item Используется для мультимедийных приложений \pause
  \item Примеры: MMX, SSE, AVX*, Neon
    \begin{itemize}
    \item 64 бит: MMX
    \item 128 бит: SSE, Neon, MSA, AltiVec
    \item 256 бит: AVX
    \item 512 бит: AVX-512
    \end{itemize}
  \end{itemize}
\end{frame}
\begin{frame}
\frametitle{WebAssembly SIMD}
  \begin{itemize}
  \item "Общий знаменатель" расширений SIMD в портебительских процессорах\footnotemark[1]
  \item SIMT и векторные операции сознательно не включены
  \item Достаточно успешно поддерживается в V8, SpiderMonkey, и нескольких автономных движках\footnotemark[2]
  \end{itemize}
  \footnotetext[1]{
    \href{https://github.com/WebAssembly/simd/blob/master/proposals/simd/WebAssembly-SIMD-May-2017.pdf}{Презентация, 05.2017}
  }
  \footnotetext[2]{
    \href{https://github.com/WebAssembly/simd/blob/master/proposals/simd/ImplementationStatus.md}{Wasm SIMD Implementation Status}
  }
\end{frame}
\begin{frame}
\frametitle{Дальнейшее развитие?}
  \begin{itemize}
  \item Процессора PC поддерживают более производительные расширения SIMD, но есть проблема с совместимостью \pause
  \item Есть примеры решения
    \begin{itemize}
    \item \href{https://github.com/google/highway}{Highway}
    \item \href{https://docs.microsoft.com/en-us/dotnet/api/system.numerics.vector}{System.Numerics.Vector} in .NET
    \end{itemize}
  \end{itemize}
\end{frame}
\begin{frame}
\frametitle{Ограничения}
  \begin{itemize}
  \item Один набор виртуальных команд на разных платформах
  \item Тривиальный выбор машинных инструкций
  \item Частичная совместимость с Wasm SIMD
  \end{itemize}
\end{frame}
\begin{frame}
\frametitle{Альтернатива -- ещё одно расширение WebAssembly SIMD}
  \begin{itemize}
  \item Более узкая поддержка в железе
  \item Плохо совместимо с общей философией выбора "общего" набора команд
  \item Сильно усложняет выбор машинных операций
  \end{itemize}
\end{frame}
\begin{frame}
\frametitle{Flexible Vectors}
  Основная идея -- операции аналогичные Wasm \textit{simd128}, но без установленной длинны вектора
  \begin{itemize}
  \item Операции с линейной памятью работают с соседними ячейками, как и в \textit{simd128}
  \item Максимальная длинна вектора устанавливается рантаймом
  \end{itemize}
\end{frame}
\begin{frame}
\frametitle{Типы данных и инструкции}
Новые типы данных и инструкции
  \begin{itemize}
  \item $vec.<type>$ -- разные типы данных для разных типов элементов
    \begin{itemize}
    \item $i8$, $i16$, $i32$, $i64$ -- целые числа
    \item $f32$, $f64$ -- действительные числа
    \end{itemize}
  \item $vec.<type>.length$ -- возвращает количество элементов в соответствующем типе
  \end{itemize}
\end{frame}
\begin{frame}
\frametitle{Типы данных и инструкции}
Расширения инструкций 
  \begin{itemize}
  \item $vec.<type>.<op>$ -- та же потоковая операция что и \textit{simd128} $<op>$, примененная к вектору длинны $vec.<type>.length$
  \item[] Например, $vec.f32.mul$ -- то же что и $f32x4.mul$ применимо к вектору длинны 4, $vec.i32.add$ -- $i32x4.add$ , и т.д.
  \end{itemize}
\end{frame}
\begin{frame}[containsverbatim]
\frametitle{Пример}
Сложение, $c = a + b$, $sz$ -- размер
\begin{lstlisting}
(block $loop
  (loop $loop_top
    (br_if $loop (i32.lt (get_local $sz) (vec.f32.length)))
    vec.f32.load (get_local $a)
    vec.f32.load (get_local $b)
    vec.f32.add
    vec.f32.store (get_local $c)
    ;; Decrement $sz and increment $a, $b, $c
    (br $loop_top)
  )
)
(block $scalar_loop ;; Finish the remaining elements
\end{lstlisting}
\end{frame}
\begin{frame}
\frametitle{Выбор машинных инструкций}
  \begin{itemize}
  \item Шаблоны аналогичные \textit{simd128}
  \item Не требует логики или сохранения глобального состояния
  \end{itemize}
\end{frame}
\begin{frame}
\frametitle{Дальнейшее расширение: векторные операции}
  Одна операция:

  \begin{itemize}
  \item $vec.<type>.set\_length$ -- установить количество элементов в соответствующем типе
  \item[] Минимум между параметром этой операции и тем что поддерживает процессор.
  \end{itemize}
\end{frame}
\begin{frame}[containsverbatim]
\frametitle{Пример векторных операций}
Vector addition, $c = a + b$, $sz$ is the size
\begin{lstlisting}
local $len i32
(block $loop
  (loop $loop_top
    (br_if $loop (i32.eq (get_local $sz) (i32.const 0)))
    (set_local $len (vec.f32.set_length (get_local $sz)))
    vec.f32.load (get_local $a)
    vec.f32.load (get_local $b)
    vec.f32.add
    vec.f32.store (get_local $c)
    ;; Decrement $sz by $len; increment $a, $b, and $c by $len
    (br $loop_top)
  )
)
\end{lstlisting}
\end{frame}
\begin{frame}
\frametitle{Дальнейшее расширение: векторные операции}
  Достоинства:
  \begin{itemize}
  \item Уменьшает размер скомпилированного модуля
  \item Может работать с машинным SIMD с поддержкой масок \pause
  \end{itemize}
  Недостатки:
  \begin{itemize}
  \item Потеря производительности на машинах без масок
  \item Изменяемое глобальное состояние
  \end{itemize}
  Пока что экспериментальное предложение.
\end{frame}
\begin{frame}
\frametitle{Текущее состояние проекта}
  \begin{enumerate}
  \item \href{https://github.com/WebAssembly/flexible-vectors}{Репозитория на Github}
  \item Начальный набор команд
  \item Первоочередная цель -- перенести операции из Wasm SIMD
  \end{enumerate}
\end{frame}
\begin{frame}
\frametitle{~}
\huge{Спасибо}
\end{frame}
\end{document}

